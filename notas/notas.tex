\documentclass{report}

% Matemática
\usepackage{amsmath}    % símbolos matemáticos
\usepackage{amsthm}     % teoremas
\usepackage{amsfonts}   % \mathbb
\usepackage{bm}         % bold math (https://ctan.org/pkg/bm)


% Figuras
\usepackage{tikz}                   % gráficos
\usepackage{float}                  % [H]
\usepackage{xcolor}                 % colores https://es.overleaf.com/learn/latex/Using_colours_in_LaTeX

% Texto
\usepackage[shortlabels]{enumitem}  % enumerate con letras


\usetikzlibrary{arrows,positioning,automata,shadows,fit,shapes}

% Teoremas, corolarios, etc.
% https://www.overleaf.com/learn/latex/theorems_and_proofs
\theoremstyle{definition} % Para que no salga en italicas

\newtheorem{theorem}{Teorema}
\newtheorem*{theorem*}{Teorema}

\newtheorem{lemma}{Lema}
\newtheorem*{lemma*}{Lema}

\newtheorem{proposition}{Prop.}
\newtheorem*{proposition*}{Prop.}

\newtheorem{definition}{Def.}
\newtheorem*{definition*}{Def}

\author{Manuel Panichelli}
\title{Notas de \\\textit{Algoritmos, Azar y Autómatas}}

\begin{document}
\maketitle

\chapter{Introducción}

\section{Azar}

Azar es \textbf{imposibilidad de predecir}, \textbf{falta de patrones},
imposibilidad de abreviar, comprimir.

Vamos a categorizar el azar según diferentes modelos de cómputo

\begin{itemize}
    \item Autómatas finitos
    \item Autómatas de pila
    \item Máquinas de turing
\end{itemize}

\begin{definition}
    Una secuencia es \textbf{azarosa} (para los autómatas de la clase $C$)
    cuando, esencialmente, la única forma de describirla (mediante un autómata
    de la clase $C$) es nombrando explícitamente cada uno de sus símbolos.
\end{definition}

Esto quiere decir que no tiene patrones (porque sino podríamos nombrar menos) y
que no se puede comprimir. \textit{Esencialmente} porque se pueden hacer
pequeñas conversiones. Por ejemplo, las cadenas de $\{a^n b^n \mid n \in
\mathbb{N}\}$ son azarosas para AF pero no para AP (porque es un lenguaje libre
de contexto pero no regular).

Hay distintos \textit{grados de azar}:

\begin{enumerate}
    \item \textbf{Azar puro}: Impredecibilidad / incompresibilidad para
    máquinas de turing
    \item \textbf{Azar básico}: Impredicibilidad / incompresibilidad para
    autómatas finitos.
\end{enumerate}

\begin{enumerate}
    \item Una secuencia es \textbf{random} si, esencialmente, sus
    \textit{segmentos iniciales} solo se pueden describir explícitamente por una
    Turing Machine (no pueden ser comprimidos por una TM)
    \item Una secuencia es \textbf{normal} si, esencialmente, sus segmentos
    iniciales solo se pueden describir explicitamente por un autómata finito.
\end{enumerate}

Cosas que no copié

\begin{enumerate}
    \item Kolmogorov / program size complexity
    \item Definicion de azar de Chaitin basado en kolmogorov
    \item Martin Löf random
\end{enumerate}

\section{Numeros normales}

\begin{definition*}
    Una \textbf{base} es un entero $\geq 2$. Para un $x \in \mathbb{R}$ en el
    intervalo unitario\footnote{El intervalo unitario es el intervalo cerrado
    $[0, 1]$}, su \textbf{expansión} en base $b$ es una \textbf{secuencia} $a_1
    a_2 a_3 \dots$ de enteros de ${0, 1, \dots, b-1}$ tales que

    $$x = 0.a_1 a_2 a_3 \dots,$$

    donde $x = \sum_{k \geq 1} \frac{a_k}{b_k}$ y $x$ no termina con una cola de
    $b - 1$.
\end{definition*}


\begin{definition}[Números normales, Borel 1909]
    Un número real $x$ es,
    \begin{itemize}
        \item \textbf{Simplemente normal a base $b$} si en la expansión de $x$
        en base $b$, cada digito ocurre con una frecuencia de $1/b$ en el
        límite.

        \textit{(En el límite todos los símbolos tienen la misma frecuencia)}
        \item \textbf{Normal a base $b$} si para cada entero positivo $k$, cada
        bloque de $k$ digitos (arrancando de cualquier posición) ocurre en la
        expansión de $x$ en base $b$ con una frecuencia en el límite de $1/b^k$
        \item \textbf{Absolutamente normal} si es normal para todas las bases.
    \end{itemize}
\end{definition}

Ejemplos:

\begin{itemize}
    \item $0.01 \ 002 \ 0003 \ 00004 \ 000005 \ 0000006 \ 00000007 \ 000000008 \dots$
    no es simplemente normal a base $10$ (el 0 tiene más frecuencia que el
    resto)
    \item $0.0123456789 \ 0.0123456789 \ 0.0123456789 \ 0.0123456789 \dots$ es
    simpelemente normal a base $10$, pero no es simplemente normal a base 100.

    \textit{Pasar de base 10 a base 100 es tomar combinaciones de dos dígitos en base 10 de forma contigua}

    \item El ternario de cantor no es simplemente normal a base 3 (las
    expansiones no tienen el dígito 1)

    \item Los numeros racionales no son normales a ninguna base
    
    Si agarro un número racional, por ej 3.14

    $$3.14 \rightsquigarrow 3.140000000\dots$$
    
    en base 10 tiene un período que se repite

    \item La constante de Liouville $\sum_{n \geq 1} 10^{-n!}$ no es normal a
    base 10
\end{itemize}

\begin{theorem}[Borel 1909]
    Casi todos los números reales son absolutamente normales.
\end{theorem}

Son las constantes matemáticas usuales como $\pi$, $e$ o $\sqrt{2}$
absolutamente normales? O al menos simplemente normales a alguna base? Es una
pregunta abierta.

\begin{theorem}[Champernowne, 1933]

    Todos los numeros naturales en base 10 concatenados es normal a base 10.

    $$0.123456789101112131415161718192021\dots$$

    \textit{No se sabe si es normal a bases que no son potencias de 10}
    
\end{theorem}

\begin{theorem}[Cassels 1959; Schmidth 1961]
    Casi todos los números del ternario de Cantor son normales a base 2.
\end{theorem}

\begin{theorem}[Bailey y Borwein 2012]
    El número de Stoneham $\alpha_{2, 3} = \sum_{k \geq 1} \frac{1}{3^k
    2^{3^k}}$ es normal a base 2 pero no simplemente normal a base 6.
\end{theorem}

\subsection{Normalidad y autómatas finitos}

\begin{definition}
    Una secuencia $x = a_1 a_2 a_3 \dots$ es \textbf{compresible} por un
    trasductor finito $T$ si y solo si en la corrida en $T$ $q_0
    \xrightarrow{a_1\mid v_1} q_1 \xrightarrow{a_2\mid v_2} q_2
    \xrightarrow{a_3\mid v_3} q_3 \dots$ satisface que

    $$\underset{n \to \infty}{\text{lim inf}}\ \frac{|v_1 v_2 \dots v_n|}{n} <
    1.$$
    
    \textit{Recordar que los $a$ son símbolols y los $v$ cadenas, posiblemente vacías.}
\end{definition}

\begin{theorem}
    Una secuencia es \textbf{normal} si y solo si es \textbf{incompresible por
    todo one-to-one transducer}.
\end{theorem}

\begin{theorem*}[Becher, Casrton, Heiber 2013]
    Los transductores finitos uno a uno no deterministicos con contadores no
    pueden comprimir secuencias normales.
\end{theorem*}

\begin{theorem*}
    
\end{theorem*}
    Los trasductores de pila no determinísticos pueden comprimir secuencias
    normales.
    
    $$
        0123456789\ \textcolor{blue}{9876543210}\
        00\ 01\ 02\ 03 \dots 98\ 99\ \textcolor{blue}{99\ 98\ 97 \dots 03\ 02\ 01\ 00\ }
        000\ 001\ 002 \dots
    $$

    \textit{Va pusheando y cuando detecta el cambio empieza a desapilar.
    Parecido al APD que reconoce $w\#w^r$}
\end{document}
