\documentclass{article}

% Matemática
\usepackage{amsmath}    % símbolos matemáticos
\usepackage{amsthm}     % teoremas
\usepackage{amsfonts}   % \mathbb
\usepackage{bm}         % bold math (https://ctan.org/pkg/bm)
\usepackage{abraces}    % \aoverbrace http://ctan.org/pkg/abraces
% https://tex.stackexchange.com/questions/132526/overbrace-and-underbrace-with-square-bracket

\usepackage[makeroom]{cancel}   % \cancel

% Figuras
\usepackage{tikz}                   % gráficos
\usepackage{float}                  % [H]
\usepackage{xcolor}                 % colores https://es.overleaf.com/learn/latex/Using_colours_in_LaTeX

% Texto
\usepackage[shortlabels]{enumitem}  % enumerate con letras

% Referencias
\usepackage[colorlinks=true]{hyperref}

\usepackage{mathtools}
% Macros para símbolos
\DeclarePairedDelimiter\ceil{\lceil}{\rceil}
\DeclarePairedDelimiter\floor{\lfloor}{\rfloor}


\newcommand{\BigO}{\mathcal{O}}

% Teoremas, corolarios, etc.
% https://www.overleaf.com/learn/latex/theorems_and_proofs
\theoremstyle{definition} % Para que no salga en italicas

\newtheorem{theorem}{Teorema}
\newtheorem*{theorem*}{Teorema}

\newtheorem{lemma}{Lema}
\newtheorem*{lemma*}{Lema}

\newtheorem{proposition}{Prop.}
\newtheorem*{proposition*}{Prop}

\newtheorem{definition}{Def.}
\newtheorem*{definition*}{Def}

\author{Manuel Panichelli}
\title{Algoritmos, Azar y Autómatas\\Resolución Ejercicios 1 a 3}

\begin{document}
\maketitle

\section*{Resultados previos}

\begin{proposition}[La cota de Gasti]\label{prop:cota-gasti}
    $\sum_{i = 1}^{n} b^i \leq b^{n+1}$
\end{proposition}
\begin{proof}[Dem.]
    \[
        \sum_{i = 1}^{n} b^i
            \leq \sum_{i = 0}^{n} b^i
            = \frac
                {b^{n+1} - 1}
                {\aunderbrace[l1r]{b - 1}_{\geq 1}}
            \leq b^{n+1} - 1 < b^{n+1}
    \]
\end{proof}

\begin{theorem}[Piatetski-Shapiro]\label{teo:piatetski-shapiro}
    Sea $x$ un número real, $b \geq 2$ un entero y $A = \{0, \dots, b - 1\}$.
    Las siguientes son equivalentes
    \begin{enumerate}
        \item $x$ es normal a base $b$
        \item Existe una constante $C$ tal que para infinitas longitudes $\ell$ y
        para todo $w \in A^\ell$

        $$
            \underset{n \to \infty}{\text{lim sup }}
            \frac{|x[1\dots n]|_w}{n} < C \cdot b^{-\ell}.
        $$
        \item Existe una constante $C$ tal que para infinitas longitudes $\ell$ y
        para todo $w \in A^\ell$

        $$
            \underset{n \to \infty}{\text{lim sup }}
            \frac{||x[1\dots n\ell]||_w}{n} < C \cdot b^{-\ell}.
        $$ 
    \end{enumerate}
\end{theorem}

\section*{Ejercicio 1}

Demostrar que el número cuya expansión decimal es

\[
    01234\dots9\ 01234\dots9\ \
    00\ 01\ 02\ \dots\ 99\ 00\ 01\ 02\ \dots\ 99 \ \
    000\ \dots\ 999\dots
\]

Es normal en base 10.

\begin{proof}[Dem.]

    Voy a llamar $C_i$ a cada concatenación de números de $i$ dígitos en base 10
    en orden lexicográfico. Por lo tanto,

    \begin{align*}
        C_1 &= 0\ 1\ 2\ 3\ \dots\ 9\\
        C_2 &= 00\ 01\ 02\ \dots\ 99 \\
        C_3 &= 000\ 001\ 002\ \dots\ 999 \\
        \vdots
    \end{align*}

    Y observo que como son progresiones aritméticas en orden lexicográfico,
    vistas como collar son collares (n-n)-perfectos, en los cuales cada bloque
    de longitud $n$ aparece exactamente $n$ veces.

    Luego la secuencia que quiero demostrar que es normal se puede escribir como

    $$x = C_1C_1\ C_2 C_2\ C_3 C_3\ \dots\ .$$

    Yo voy a querer contar cantidad de apariciones, y se que en
    \textit{collares} perfectos es exactamente $n$, pero esto son secuencias.
    Para tomarlas como secuencias de todas formas,

    \begin{itemize}
        \item En el primer bloque $C_i$ de $C_iC_i$, puedo tomar los primeros $i
        - 1$ del 2do $C_i$ para leer el primero como un collar.
        \item En el segundo bloque, voy a poder tomar los primeros $i - 1$ del
        bloque de siguiente orden $C_iC_iC_{i+1}$. Esto es porque al ser
        progresiones aritméticas, en la $i+1$-ésima seguro los primeros $i - 1$
        dígitos son iguales a los primeros $i - 1$ de la $C_i$, ya que están
        ordenadas lexicográficamente.
    \end{itemize}

    Defino $S_k$ como la longitud del segmento inicial que contiene hasta la
    aparición completa de $C_k C_k$. Se que $|C_k| = k \cdot b^k$ pues no es más
    que la concatenación de todos los bloques posibles de longitud $k$. Luego,

    \[S_k = \sum_{i=1}^n 2|C_i| = 2\sum_{i=1}^n 2i \cdot b^i \]

    sean $b = |A| = 10$, $u \in A^\ell$ un bloque de longitud $\ell$. Tomo $n
    \geq S_\ell$. Se que la cantidad de apariciones de $u$ en el bloque $C_nC_n$
    y los $n-1$ siguientes correspondientes a los primeros $n-1$ del bloque
    $C_{n+1}$ (para que el 2do bloque $C_n$ pueda ser tomado como collar)
    son menores o iguales a

    \[
        2n\cdot b^{n-\ell} + n - \ell.
    \]

    Pues,

    \begin{itemize}
        \item Teniendo en cuenta la observación anterior, podemos tomarlos como
        collar. Por lo tanto, sabemos que un collar $(n, n)$ aparecerá $n$ veces
        y como tengo dos concatenados será $2n$.
        \item $b^{n-\ell}$ es la cantidad de rellenos posibles para bloques de
        tamaño $n$ que comiencen con $u$.
        \item $n - \ell$ son las posiciones en las cuales podría aparecer un
        bloque $u$ que no pueden ser el principio de un bloque de tamaño $n$,
        porque se pasarían del tamaño de la secuencia.
    \end{itemize}

    Sea $N$ una posición cualquiera. Tomo $n$ tal que

    \[
        S_n \leq N < S_{n+1}.
    \]

    Para poder usar el Teorema de \nameref{teo:piatetski-shapiro}, voy a tener
    que acotar el cociente $\frac{|x[1\dots N]|_u}{N}$. Noto que

    \begin{align*}
        |x[1\dots N]|_u
            &\leq |x[1\dots S_{n+1}+n]|_u\\
            &\leq \aunderbrace[l1r]{
                |x[1\dots S_{\ell} - 1]|_u + 
                \aoverbrace[L1R]{\ell - 1}
                    ^{\hidewidth\text{entre bloques}\hidewidth}}_{A}
                +|x[S_{\ell}\dots S_{n+1}+n]|_u\\
            &\leq A + \sum_{i=\ell}^{n+1} |x[S_{i-1} \dots S_i+(i-1)]|_u + \aoverbrace[L1R]{\ell - 1}^{\hidewidth\text{entre bloques}\hidewidth} \\
            &= A + \sum_{i=\ell}^{n+1} 2ib^{i - \ell} + i - \cancel{\ell} + \cancel{\ell} - 1 \\
            &= A + b^{-\ell}\sum_{i=\ell}^{n+1} 2(n+1) b^{i} + 
                \aunderbrace[l1r]{\sum_{i=\ell}^{n+1} i - 1}_{B}\\
            &\leq A + B + 2(n+1) b^{-\ell} b^{n+2} & \text{(Por \ref{prop:cota-gasti})}
    \end{align*}

    Luego,
    \begin{align*}
        \limsup_{N \to \infty} \frac{|x[1\dots N]|_u}{N}
            &\leq \limsup_{N \to \infty} \frac{A + B + 2(n+1) b^{-\ell} b^{n+2}}{S_n} \\
            &= \limsup_{N \to \infty} \cancelto{0}{\frac{A}{S_n}} +
                \cancelto{0}{\frac{B}{S_n}} +
                \frac{2(n+1) b^{-\ell} b^{n+2}}{\sum_{i=1}^{n}2ib^i} \\
            &\leq \limsup_{N \to \infty} \frac{2(n+1) b^{-\ell} b^{\cancel{n}+2}}{2n\cancel{b^n}} \\
            &= \limsup_{N \to \infty} \frac{2n+2}{2n} b^{-\ell} b^{2} \\
            &= \limsup_{N \to \infty} (1 + \cancelto{0}{\frac{1}{n}}) b^{-\ell} b^{2} \\
            &= b^2 \cdot b^{-\ell} \\
            &< \aunderbrace[l1r]{b^3}_{C} \cdot b^{-\ell}
    \end{align*}

    Por lo tanto, por el Teorema de \nameref{teo:piatetski-shapiro} con $C =
    b^3$ la secuencia $x$ es normal a base 10.
\end{proof}

\section*{Ejercicio 2}

Demostrar que la concatenación de secuencias de Bruijn de orden sucesivo es
normal.

\begin{proof}[Dem.]
    Voy a llamar $C_k$ a cada secuencia de de Bruijn de orden $k$. Luego, puedo
    expresar la cadena como,

    \[
        x = C_1 C_2 C_3 \dots\ .
    \]

    Teniendo en cuenta que la longitud de una secuencia de Bruijn de orden $k$
    es $b^k + k - 1$, defino $S_k$ como la longitud del segmento inicial de $x$
    que contiene hasta $C_k$ inclusive:

    \[
        S_k = \sum_{i=1}^{k} b^i + i - 1.
    \]

    Sea $\ell$ una longitud, $u \in A^\ell$ un bloque de esa longitud y tomo
    algún $n > S_\ell$ (para que seguro esté la secuencia de Bruijn que contiene
    a $u$). Como la cantidad de apariciones de $u$ en $C_\ell$ es exactamente
    una, la cantidad de apariciones en $C_n$ es

    \[
        b^{n-\ell} + n - \ell,
    \]

    Ya que

    \begin{itemize}
        \item $b^{n-\ell}$ es la cantidad de formas posibles de
        \textit{rellenar} palabras de longitud $n$ que arranquen con $u$.
        \item $n - \ell$ contempla todas las palabras que no podrían comenzar
        por $u$ porque se pasan del final. 
    \end{itemize}

    Luego, en la concatenación de todas las de Bruijn, puedo contar en cada una
    y luego contemplar las $\ell - 1$ posiciones en las que podría aparecer
    entre cada una.

    Sea $N$ una posición cualquiera, y tomo $n$ tal que

    $$S_{n} \leq N \leq S_{n+1}.$$

    Quiero usar el Teorema de \nameref{teo:piatetski-shapiro}, por lo que voy a
    tener que acotar de alguna forma el cociente $\frac{|x[1\dots N]|_u}{N}$.
    Veamos el numerador,

    \begin{align*}
        |x[1\dots N]|_u
            &\leq |x[1\dots S_{n+1}]|_u\\
            &\leq \aunderbrace[l1r]{
                |x[1\dots S_{\ell} - 1]|_u + 
                \aoverbrace[L1R]{\ell - 1}
                    ^{\hidewidth\text{entre bloques}\hidewidth}}_{A}
                +|x[S_{\ell}\dots S_{n+1}]|_u\\
            &\leq A + \sum_{i=\ell}^{n+1} |x[S_{i-1} \dots S_i]|_u + \aoverbrace[L1R]{\ell - 1}^{\hidewidth\text{entre bloques}\hidewidth} \\
            &= A + \sum_{i=\ell}^{n+1} b^{i - \ell} + i - \cancel{\ell} + \cancel{\ell} - 1 \\
            &= A + b^{-\ell}\sum_{i=\ell}^{n+1} b^{i} + 
                \aunderbrace[l1r]{\sum_{i=\ell}^{n+1} i - 1}_{B}\\
            &\leq A + B + b^{-\ell} b^{n+2} & \text{(Por \ref{prop:cota-gasti})}
    \end{align*}

    Ahora veamos el cociente completo,

    \begin{align*}
        \frac{|x[1\dots N]|_u}{N}
            &\leq \frac{A + B + b^{-\ell} b^{n+2}}{S_n} \\
            &= \frac{A}{S_n} + \frac{B}{S_n} + \frac{ b^{-\ell} b^{n+2}}{S_n}
    \end{align*}

    Veamos los cocientes por separado. Primero, como $A$ es una constante que
    solo depende de $\ell$, claramente $\frac{A}{S_n} \xrightarrow[N \to
    \infty]{} 0$. Además,

    \begin{align*}
        \frac{B}{S_n}
            &= \frac{\sum_{i=\ell}^{n+1} i - 1}{\sum_{i=1}^{n}b^i + i - 1}\\
            &\leq \frac{\frac{n(n+1)}{2}}{b^n + n - 1}\\
            &\approx \frac{\BigO(n^2)}{\BigO(b^n)} \xrightarrow[N \to \infty]{} 0.
    \end{align*}

    Finalmente,

    \begin{align*}
        \frac{b^-\ell b^{n+2}}{S_n}
            &= \frac{b^-\ell b^{n+2}}{\sum_{i=1}^{n}b^i + i - 1} \\
            &\leq \frac{b^-\ell b^{n+2}}{b^n + \aunderbrace[l1r]{n - 1}_{\geq 0}} \\
            &\leq \frac{b^-\ell b^{\cancel{n}+2}}{\cancel{b^n}}\\
            &= b^{-\ell} b^2.
    \end{align*}

    Juntando todo,

    \begin{align*}
        \limsup_{N \to \infty} \frac{|x[1\dots N]|_u}{N}
            &\leq \cancelto{0}{\frac{A}{S_n}} + \cancelto{0}{\frac{B}{S_n}} + b^{-\ell} b^2 \\
            &< b^{-\ell}  \aunderbrace[l1r]{b^3}_{C}.
    \end{align*}

    En conclusión, por el Teorema de \nameref{teo:piatetski-shapiro} tomando $C
    = b^3$, $x$ es normal para cualquier base $b$, y por lo tanto es normal.
\end{proof}

\section*{Ejercicio 3}

\begin{definition}[Collares $(k, n)$-perfectos] Una secuencia circular es $(k,
n)$-perfecta si tiene longitud $n |A|^k$ y cada palabra de longitud $k$ ocurre
exactamente $n$ veces en posiciones diferentes modulo $n$, para cualquier
convención del punto inicial.

\begin{definition}[Secuencias $(k, k)$-perfectas] Una secuencia (no circular)
perfecta de orden $k$ se obtiene partiendo de una secuencia circular $(k,
k)$-perfecta y agregando como sufijo los primeros $k-1$ símbolos (así simulando
la circularidad)
\end{definition}
    
\end{definition}

Demostrar que la concatenación de secuencias $(2n, 2n)$-perfectas para
$n = 1, 2, 3, \dots$ es una secuencia normal.

\begin{proof}[Dem.]
Voy a llamar a cada secuencia $(2i, 2i)$-perfecta $C_i$, y de esa forma mi
secuencia entera se puede expresar como

$$x = C_1 C_2 \dots.$$

Tomo $b = |A|$ el tamaño del alfabeto, y defino la longitud del segmento inicial
hasta la secuencia $C_i$ inclusive como

\begin{align*}
    S_k &= \sum_{i=1}^{k} |C_i| \\
        &= \sum_{i=1}^{k} 2i \cdot b^{2i} + 2i - 1.
\end{align*}

Para ver que sea normal, quiero aplicar el Teorema de
\nameref{teo:piatetski-shapiro}. Sea $\ell$ una longitud, $u \in A^\ell$ un
bloque. Tomo $n > S_{\ceil{\ell/2}}$. Se que en una secuencia $(2n,
2n)$-perfecta, $u$ aparece a lo sumo $2n \cdot b^{2n -\ell} + 2n - \ell$ veces,
pues

\begin{itemize}
    \item $2n$ es la cantidad de apariciones de cualquier palabra de longitud
    $2n$ en un collar $(2n, 2n)$-perfecto, por definición
    \item $b^{2n -\ell}$ es la cantidad de formas diferentes de
    \textit{rellenar} las palabras que comiencen con el bloque $u$.
    \item $2n - \ell$ contempla las posiciones en las cuales podría aparecer $u$
    que no pueda a ser al principio de una palabra de longitud $2n$, porque se
    pasaría del final del bloque.
\end{itemize}

% TODO: Dibujito

Sea $N$ una posición cualquiera de $x$. Puedo tomar $n$ tal que

\[S_n \leq N < S_{n+1}.\]

Luego,

\begin{align*}
    |x[1\dots N]|_u
        &\leq |x[1\dots S_{n+1}]|_u \\
        &\leq 
            \aunderbrace[l1r]{|x[1\dots \ceil{\ell/2} - 1]|_u
            + \aoverbrace[L1R]{\ell - 1}^{\hidewidth\text{entre bloques}\hidewidth}}_{A}
            + |x[\ceil{\ell/2} \dots S_{n+1}]|_u \\
        & = A + |x[\ceil{\ell/2} \dots S_{n+1}]|_u \\
        &\leq A +
            \sum_{i = \ceil{\ell/2}}^{n+1}
                |x[S_{i-1}\dots S_{i}]|_u
                + \aoverbrace[L1R]{\ell - 1}^{\hidewidth\text{entre bloques}\hidewidth} \\
        &\leq A +
            \sum_{i = \ceil{\ell/2}}^{n+1}
                2i \cdot b^{2i - \ell} + 2i - \bcancel{\ell}
                + \bcancel{\ell} - 1 \\
        &= A +
            \sum_{i = \ceil{\ell/2}}^{n+1} 2i \cdot b^{2i - \ell}
            + \aoverbrace[L1R]{\sum_{i = \ceil{\ell/2}}^{n+1} 2i - 1}^{B} \\
        &\leq A + B +
            \sum_{i = \ceil{\ell/2}}^{n+1} 2(n+1) \cdot b^{2i - \ell} \\
        &= A + B +
            2(n+1) b^{-\ell}\sum_{i = \ceil{\ell/2}}^{n+1} b^{2i}\\
        &\leq A + B +
            2(n+1) b^{-\ell} b^{2(n+2)} &\text{(Por Prop. \ref{prop:cota-gasti})}
\end{align*}

Por lo tanto,

\begin{align*}
    \frac{|x[1\dots N]|_u}{N}
        &\leq \frac{|x[1\dots S_{n+1}]|_u}{S_n} \\
        &\leq \frac{A + B + 2(n+1) b^{-\ell} b^{2(n+2)}}{S_n} \\
        &= \frac{A}{S_n} + \frac{B}{S_n} + \frac{2(n+1) b^{-\ell} b^{2(n+2)}}{S_n}
\end{align*}

Veamos cada una por separado. Primero,

\begin{align*}
    \frac{2n b^{-\ell} b^{2(n+2)}}{S_n}
        &= \frac
            {2(n+1) b^{-\ell} b^{2(n+2)}}
            {\sum_{i=1}^{n} 2i \cdot b^{2i} + 2i - 1} \\
        &\leq \frac
            {(2n + 2) b^{-\ell} b^{2(n+2)}}
            {2n \cdot b^{2n} + \aunderbrace[l1r]{2n - 1}_{\geq 0}} \\
        &\leq \frac
            {(2n+2) b^{-\ell} b^{\bcancel{2n}+4}}
            {2n \cdot \bcancel{b^{2n}}}\\
        &=\frac{\cancel{2n}b^{4-\ell}}{\cancel{2n}} +
        \aunderbrace[l1r]{\frac{2b^{4-\ell}}{2n}}_{\xrightarrow[N \to \infty]{}\ 0}
        \\ 
        &= b^{-\ell} b^{4}
\end{align*}

Como $A$ no depende de $n$, claramente $\frac{A}{S_n} \underset{N \to
\infty}{\longrightarrow} 0$. Además, como
$$B = \sum_{i = \ceil{\ell/2}}^{n+1} 2i - 1 \leq \sum_{i = 1}^{n+1} 2i - 1 = n^2,$$

Tengo que

\begin{align*}
    \frac{B}{S_n}
        &= \frac
            {\sum_{i = \ceil{\ell/2}}^{n+1} 2i - 1}
            {\sum_{i=1}^{n} 2i \cdot b^{2i} + 2i - 1} \\
        &\leq \frac
            {n^{\cancel{2}}}
            {2\cancel{n} \cdot b^{2n}}\\
        &\approx \frac{\BigO(n)}{\BigO(b^n)}
        \underset{N \to
        \infty}{\longrightarrow} 0
\end{align*}

Juntando todo,

\begin{align*}
    \limsup_{N \to \infty} \frac{|x[1\dots N]|_u}{N}
        &\leq \limsup_{N \to \infty}
            \cancelto{0}{\frac{A}{S_n}} +
            \cancelto{0}{\frac{B}{S_n}} +
            \frac{2(n+1) b^{-\ell} b^{2(n+2)}}{S_n} \\
        &= b^{-\ell}b^4 < \aunderbrace[l1r]{b^5}_{C} \cdot b^{-\ell}.
\end{align*}

Por lo tanto, aplicando el Teorema de \nameref{teo:piatetski-shapiro} con $C =
b^5$ afirmamos que $x$ es normal para todo $b$, entonces es normal.

\end{proof}

\end{document}