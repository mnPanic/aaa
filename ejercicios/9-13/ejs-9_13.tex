\documentclass{article}

% https://es.overleaf.com/learn/latex/Page_size_and_margins
\usepackage[a4paper]{geometry}

% Matemática
\usepackage{amsmath}    % símbolos matemáticos
\usepackage{amsthm}     % teoremas
\usepackage{amsfonts}   % \mathbb
\usepackage{amssymb}    % \therefore
\usepackage{bm}         % bold math (https://ctan.org/pkg/bm)
\usepackage{abraces}    % \aoverbrace http://ctan.org/pkg/abraces
% https://tex.stackexchange.com/questions/132526/overbrace-and-underbrace-with-square-bracket

\usepackage[makeroom]{cancel}   % \cancel

% Figuras
\usepackage{tikz}                   % gráficos
\usepackage{float}                  % [H]
\usepackage{xcolor}                 % colores https://es.overleaf.com/learn/latex/Using_colours_in_LaTeX

% Formateo
\usepackage{framed}     % env leftbar

% Texto
\usepackage[shortlabels]{enumitem}  % enumerate con letras

% Referencias
\usepackage[colorlinks=true]{hyperref}

% Código
% https://tex.stackexchange.com/questions/51295/algpseudocode-without-end-block-text
\usepackage[noend]{algpseudocode} % env algorithmic,
\usepackage{algorithm}
\usepackage{algorithmicx}


% Diagramas
\usepackage{tikz}
\usetikzlibrary{automata, arrows, positioning}

\tikzset{
    ->, % makes the edges directed
    >=stealth', % makes the arrow heads bold
    node distance=3cm, % specifies the minimum distance between two nodes. Change if necessary.
    every state/.style={thick}, % sets the properties for each 'state' node
    initial text=$ $, % sets the text that appears on the start arrow
}


\usepackage{mathtools}
% Macros para símbolos
\DeclarePairedDelimiter\ceil{\lceil}{\rceil}
\DeclarePairedDelimiter\floor{\lfloor}{\rfloor}

\newcommand{\BigO}{\mathcal{O}}

% Teoremas, corolarios, etc.
% https://www.overleaf.com/learn/latex/theorems_and_proofs
\theoremstyle{definition} % Para que no salga en italicas

\newtheorem{theorem}{Teorema}
\newtheorem*{theorem*}{Teorema}

\newtheorem{lemma}{Lema}
\newtheorem*{lemma*}{Lema}

\newtheorem{proposition}{Prop.}
\newtheorem*{proposition*}{Prop}

\newtheorem{corollary}{Corolario}
\newtheorem*{corollary*}{Corolario}

\newtheorem{definition}{Def.}
\newtheorem*{definition*}{Def}

\newtheorem*{observation*}{Obs}

\newtheorem{example}{Ejemplo}
\newtheorem*{example*}{Ejemplo}

% Comandos de AAA
\newcommand{\select}{\upharpoonright}
\newcommand{\progends}{\downarrow}
\newcommand{\firstn}[1]{\first{n}{#1}}
\newcommand{\first}[2]{#2[1 \dots #1]}
\newcommand{\measure}[1]{\mu\left(#1\right)}
\newcommand{\size}[1]{\left|#1\right|}
\newcommand{\sucML}{(V_i)_{i > 0}}
\newcommand{\ZOalphabet}{\{0, 1\}^*}
\newcommand{\ML}{Martin-Löf }
\newcommand{\occurrencesOf}[2]{|#2|_#1}
\newcommand{\Bad}[4]{\text{Bad}(#1, #2, #3, #4)}


% Entornos
\newenvironment{nota}[1]
    {\begin{leftbar}\textbf{#1}}
    {\end{leftbar}}

\author{Manuel Panichelli}
\title{Algoritmos, Azar y Autómatas\\Resolución Ejercicios 9 a 13 (Aleatoriedad)}

\begin{document}
\maketitle

\section*{Resultados previos}

\begin{definition}[Programa elegante]\label{def:prog-elegante}
    Un programa es elegante si es el más corto que computa una secuencia. Es
    decir, $p \in \ZOalphabet$ es un programa elegante si existe $s \in
    \ZOalphabet$, $U(p) = s$ y para todo otro programa $p \in \ZOalphabet$, si
    $U(p') = s$ entonces $|p| \leq |p'|$.

    Llamamos $s^*$ al programa elegante que computa $s$, $U(s^*) = s$ y $K(s) = |s^*|$.
\end{definition}

\begin{definition}[Abiertos básicos]
    $B_s$ es el conjunto de palabras infinitas que comienzan por $s$.

    \[B_s = \{ s \alpha \mid \alpha \in \{0, 1\}^* \}.\]

\begin{proposition}[Medida de un abierto]
    $\measure{B_s} = 2^{-|s|}.$
\end{proposition}
\end{definition}

\begin{definition}[Test de Martin-Löf]\label{def:martin-lof}
    Una sucesión $\sucML$ es un test de Martin-Löf si

    \begin{itemize}
        \item $V_i = \bigcup_{s \in S_i} B_s$ con $S_1 \subseteq {0, 1}^*$
        computable
        \item $\measure{V_i} \leq 2^{-i}$ \textit{(se va achicando)}
    \end{itemize}

    Si existe un test tal que $x \in \bigcap_{i \geq 1} V_i$, entonces $x$ no es
    aleatorio.
\end{definition}

\newpage

\section*{Ejercicio 9}

\textit{Demostrar que los números \ML aleatorios son normales.}

Voy a demostrarlo por el contrarecíproco:

\[
    x \text{ no es normal } \Rightarrow x \text{ no es \ML aleatorio. }
\]

Primero listo algunas definiciones y propiedades que voy a necesitar solo para
este ejercicio,

\begin{definition}[Simple normalidad en una base]\label{def:simple-norm}
    $x \in A^*$ es simplemente normal en base $b$ si para todo $w \in A^\ell$

    \[
        \lim_{n\to\infty}
            \frac{\occurrencesOf{w}{\first{n}{x}}}{n}
            = \frac{1}{b^\ell}.
    \]
\end{definition}

\begin{definition}[Malas palabras]\label{def:bad}
    El conjunto de malas palabras para un alfabeto $A$ es

    \[
        \Bad{A}{k}{w}{\varepsilon} = \left\{
            v \in A^k : \left|
                \occurrencesOf{w}{v} - \frac{k}{b^{|w|}}
            \right| > \varepsilon k
        \right\}.
    \]
\end{definition}

\begin{proposition}[Cota de cardinalidad de Bad]\label{prop:bad}
    Para $k$ suficientemente grande,
    \[
        \size{\Bad{A}{k}{w}{\varepsilon}} < \varepsilon b^k.
    \]
\end{proposition}

\begin{proposition}[Suma geométrica infinita]\label{prop:geom-inf-sum}
    Sea $-1 < r < 1$, $r \neq 0$,

    \[
        \sum_{k=0}^{\infty} ar^k
        = \frac{a}{1 - r}.
    \]
\end{proposition}

\begin{proof}[Dem]
    Sea $x$ no normal. Para ver que no es \ML aleatorio tenemos que dar un
    \nameref{def:martin-lof} $\sucML$ tal que $x \in \bigcap_{i \geq 1} V_i$.

    Como no es normal, existe una base $b$ tal que $x$ no es normal en esa
    base.

    \begin{align*}
        &\Rightarrow
            \exists\ w \text{ palabra tal que }
            \lim_{n\to\infty}
                \frac{\occurrencesOf{w}{\first{n}{x}}}{n}
            \neq b^{-|w|}
        &&\text{(Def \ref{def:simple-norm})}\\
        &\Rightarrow
            \exists\ w, \delta, k_0 \text{ tal que }
                \forall k > k_0,\
                    |\occurrencesOf{w}{\first{k}{x}} - b^{-|w|}| > \delta\\
        &\Rightarrow
            \exists\ w, \delta, k_0 \text{ tal que }
            \forall k > k_0,\
                \first{k}{x} \in \Bad{A}{k}{w}{\delta},
        &&\text{(Def \ref{def:bad})}
    \end{align*}

    y observo que con $\delta$ fijo, $\forall \varepsilon \leq \delta$

    \[
        \Bad{A}{k}{w}{\delta} \subseteq \Bad{A}{k}{w}{\varepsilon}
    \]

    ya que si con una cota mas laxa una palabra resulta mala, con una condición
    más estricta también lo será.

    Definimos el conjunto $S(i)$

    \begin{align*}
        S(i)
            &= \bigcup_{k > f(i)} 
                \bigcup_{u : |u| < g(i)}
                    \Bad{A}{k}{u}{\varepsilon(k)}\\
            &=
                \begin{aligned}[t]
                &\text{palabras de longitud} > f(i)
                \text{ con mala frecuencia}\\
                &\text{con respecto a palabras de longitud} < g(i),
                \end{aligned}
    \end{align*}

    con $f, g$ crecientes y $\varepsilon$ decreciente. Definimos también el
    conjunto $V_i$ a partir de él,

    \[
        V_i = \bigcup_{s \in S(i)} B_s
    \]

    que es casi el test pero podría suceder que $x$ no pertenezca. Para
    solucionarlo, vamos a \textit{shiftearlo}. Como $x$ es mala, sabemos que
    $\first{k}{x} \in \Bad{A}{k}{w}{\delta}$. Luego, como $f$ y $g$ son
    crecientes, existe un $i_0$ a partir del cual
    \begin{itemize}
        \item $f(i_0) > k_0$ (estamos considerando palabras lo suficientemente
        grandes), y
        \item $g(i_0) > |w|$ (estamos considerando longitudes lo suficientemente
        grandes).
    \end{itemize}

    y por lo tanto, $\forall i > i_0$, $x \in V_{i+i_0}$. Definimos entonces el test a partir de ese $i_0$,

    \[
        W_i = V_{i_0 + i}.
    \]

    Tiene que cumplir dos cosas,

    \begin{enumerate}
        \item $x \in \bigcap_{i > 0} W_i$ y
        \item $\measure{W_i} = \measure{V_i} \leq 2^{-i}$.
    \end{enumerate}

    Como sabemos que 1. se cumple por construcción de $W_i$ y la elección de
    $i_0$, nos queda elegir $f, g$ y $\varepsilon$ de forma tal que se cumpla 2.
    Veamos la medida de cada conjunto,

    \begin{align*}
        \measure{W_i}
            &= \measure{V_i}\\
            &= \measure{\bigcup_{s\in S(i)} B_s}\\
            &\leq \sum_{s \in S(i)} \measure{B_s}\\
            &= \sum_{s \in S(i)} 2^{-|s|}\\
            &< \sum_{s \in S(i)} 2^{-f(i)}\\
            &=\size{S(i)} 2^{-f(i)} < 2^{-i} \iff \size{S(i)} < 2^{f(i) - i}.\\
    \end{align*}

    Para $k$ suficientemente grande, se que se cumple la Prop \ref{prop:bad}

    \begin{align*}
        \size{S(i)}
            &= \size{
                \bigcup_{k > f(i)} 
                \bigcup_{u : |u| < g(i)}
                \Bad{A}{k}{u}{\varepsilon(k)}
            }\\
            &\leq 
                \sum_{k > f(i)} 
                \sum_{u : |u| < g(i)}
                \size{\Bad{A}{k}{u}{\varepsilon(k)}}\\
            &<
                \sum_{k > f(i)} 
                \sum_{u : |u| < g(i)}
                    \varepsilon(k) 2^k
            &&(\text{Prop. \ref{prop:bad}})\\
            &<
                \sum_{k > f(i)} 
                    g(i)
                    \varepsilon(k) 2^k\\
            &< \sum_{k > 0} 
                    g(i)
                    \varepsilon(k) 2^k\\
            &= \sum_{k > 0} 
                g(i)
                (1/2)^k \cancel{2^{-k}} \cancel{2^k}
            &&(\text{Tomando } \varepsilon(k) = (1/2)^k 2^{-k})\\
            &= \frac
                {g(i)}
                {1 - 1/2}
            = 2g(i)
            &&\text{(Prop. \ref{prop:geom-inf-sum})}
    \end{align*}

    Juntando, quiero encontrar $g$ y $f$ tales que $2g(i) < 2^{f(i)- i}$.
    Si tomo $f(i) = 2i$ y $g(i) = i/2$, tengo que

    \begin{align*}
        2g(i) < 2^{f(i)- i}
            &\iff 2 i/2 < 2^{2i - i} \\
            &\iff i < 2^{i},
    \end{align*}

    lo cual es cierto. Por lo tanto, las siguientes elecciones de funciones
    hacen que el test $W_i$ cumpla con la restricción 2,
    
    \begin{itemize}
        \item $f(n) = 2n$ que es creciente
        \item $g(n) = n/2$ que es creciente
        \item $\varepsilon(n) = (1/2)^n 2^{-n} = (1/4)^n$ que es decreciente.
    \end{itemize}

    Finalmente, como existe un test de \ML que contiene a $x$, no es aleatoria.
    Por lo tanto, como acabo de demostrar que cuando $x$ no es normal tampoco es
    aleatorio, concluyo que los números \ML aleatorios son normales.
\end{proof}

\newpage 

\section*{Ejercicio 10}

\textit{Dar un algoritmo que permite computar $\Omega$ con un oráculo para el problema de la detención.}

Recuerdo la definición de $\Omega$, la suma de las potencias de 2 de las
longitudes de todos los programas que terminan. Si U es una máquina de Turing
universal,

\[
    \Omega = \sum_{U(p)\progends} 2 ^ {-|p|}
\]

para computar $\Omega$ voy a dar una función que computa los primeros $n$
dígitos, $\firstn{\Omega}$, y luego voy incrementando $n$ e imprimiendo los
resultados.

Sea $g$ una enumeración de todos los programas que terminan (se
podría obtener computablemente mediante un método como \textit{dovetailing}),
defino una aproximación de $\Omega$ hasta el m-ésimo programa,

\[
    \alpha_m = \sum_{j = 1}^{m} 2 ^ {-|g(j)|}.
\]

Lo que me gustaría saber es para qué $m$ $\alpha_m$ tiene los primeros $n$
dígitos \textit{definitivos}, es decir $\firstn{\Omega} = \firstn{\alpha_m}$.
Para ello, debo verificar que

\[
    \firstn{\alpha_m} \overset{?}{=} \firstn{\alpha_m'}\ \forall m' > m.
\]

Es decir, no importa que sigamos considerando más programas, los primeros $n$
dígitos no van a cambiar. Como no es algo finito, no lo podemos computar con un
algoritmo, pero acá es donde nos salva el oráculo de Halt. La siguiente función
logra lo buscado cuando se la pasamos al oráculo:

\vspace{0.5cm}

\begin{algorithmic}
    \Function{Q}{m, n}
        \State $m' \gets m + 1$
        \While{true}
            \If{$\firstn{\alpha_m} \neq \firstn{\alpha_m'}$}
            \State \textbf{break}
            \EndIf
            \State $m' \gets m' + 1$
        \EndWhile
    \EndFunction
\end{algorithmic}

\begin{itemize}
    \item Si \texttt{OraculoHalt(Q(m, n)) = true} (es decir, Q termina) es
    porque existía $m$ tal que cambiaban los primeros $n$ dígitos, y por lo
    tanto no eran definitivos.

    \item Si \texttt{OraculoHalt(Q(m, n)) = false} (es decir, Q \textit{no}
    termina), entonces no existe $m$ tal que cambien los primeros $n$ dígitos, y
    por lo tanto son definitivos.
\end{itemize}

El algoritmo completo para computar $\Omega$ es el siguiente, donde la función
sin argumentos Print $\Omega$ lo imprime de a segmentos iniciales.

\vspace{0.5cm}

\begin{algorithmic}
    \Function{Print $\Omega$}{}

        \For{$n = 1, 2 \dots$}
            \State \textbf{print} \Call{$\Omega$}{n}
        \EndFor

    \EndFunction

    \Function{$\Omega$}{n}

    \For{$m = 1, 2 \dots$}
        \If{$\neg$ OraculoHalt(\Call{Q}{m, n})}
            \State \textbf{return} $\firstn{\alpha_m}$
        \EndIf
    \EndFor

\EndFunction
    \Function{Q}{m, n}
        \State $m' \gets m + 1$
        \While{true}
            \If{$\firstn{\alpha_m} \neq \firstn{\alpha_m'}$}
            \State \textbf{break}
            \EndIf
            \State $m' \gets m' + 1$
        \EndWhile
    \EndFunction
\end{algorithmic}

\section*{Ejercicio 11}

\textit{Salteado porque no era necesario resolverlo.}

\section*{Ejercicio 12}

\subsection*{12.1}

\textit{Demostrar que el número $(1 - \Omega)$ es aleatorio}

\begin{lemma}\label{lemma:xor-bits}
    Con los bits de $x$ puedo obtener los de $\bar{x} = 1 - x$ realizando un xor con
    todos 1s, $x \oplus 1s = \bar{x}$.
\end{lemma}

\begin{proof}[Dem]
    Sea $g: \mathbb{N} \to \Sigma^*$ una enumeración de los programas que se
    detienen, $\alpha_m = \sum_{j = 1}^{m} 2 ^ {-|g(j)|}$ una aproximación de
    $\Omega$ de $m$ pasos. Defino el programa $p$,

    \[
        p = b_1 b_2 \dots b_c \first{i}{\bar{\Omega}}^*
    \]

    donde $\first{i}{\bar{\Omega}}^*$ es una subrutina que es el programa
    elegante (el más corto) que computa $\first{i}{\bar{\Omega}}$ y $b_1 b_2
    \dots b_c$ realiza los siguientes pasos,

    \begin{enumerate}
        \item[0.] Computamos $\first{i}{\bar{\Omega}}$ con
        $\first{i}{\bar{\Omega}}^*$.
        \item Computamos $\first{i}{\Omega}$ en base a
        $\first{i}{\bar{\Omega}}$ mediante un $\oplus$ con todos 1s.
        \item $m = 1$. Mientras ($\alpha_m \leq \first{i}{\Omega}$), $m = m + 1$.
        \item Sea $O = \{ U(g(j)) \mid 1 \leq j \leq m \}$ los outputs de los
        programas que se detienen que aportan a la aproximación de $\Omega$.
        \item Sea $s$ la cadena más chica lexicográficamente tal que $s \notin O$.
        \item Return $s$.
    \end{enumerate}

    Veamos que $|s^*| > i$, la longitud del programa elegante que computa $s$
    (su complejidad) es más chica que $i$. Para demostrarlo supongamos lo
    contrario, que $|s^*| \leq i$

    \begin{align*}
        \Omega 
        &> 2^{-|s^*|} + \alpha_m
            &&\text{($\Omega$ tiene infinitos aportes)}\\
        &> 2^{-|s^*|} + \first{i}{\Omega}
            &&\text{(porque $\alpha_m > \first{i}{\Omega}$)}\\
        &\geq 2^{-i} + \first{i}{\Omega}
            &&\text{(sup. $|s^*| \leq i$)}\\
        &\geq \Omega.
            &&\text{($\first{i}{\Omega}$ contiene exactamente los primeros $i$ bits de $\Omega$)}
    \end{align*}

    Y llegamos a $\Omega > \Omega$ que es un absurdo. Por lo tanto, $|s^*| > i$.
    
    Por otro lado, como $p$ es un programa (no muy bueno) que computa $s$, se que

    \begin{equation}\label{eq:12-comp}
         K(s) \leq c + |\first{i}{\bar{\Omega}}^*|
    \end{equation}

    donde $c = |b_1 \dots b_c|$.

    Juntando,

    \begin{align*}
        i
            &< |s^*|\\
            &= K(s)\\
            &\leq c + |\first{i}{\bar{\Omega}}^*|
            &&\text{por (\ref{eq:12-comp})}\\
            &= c + K(\first{i}{\bar{\Omega}})
    \end{align*}

    $\iff K(\first{i}{\bar{\Omega}}) > i - c$, que es la definición de
    aleatoriedad de Chaitin. Concluyo que $\bar{\Omega}$ es aleatorio.

\end{proof}

\subsection*{12.2}

\textit{Para todo conjunto $X$ infinito y c.e pero no computable, $\alpha =
\sum_{x \in X} 2^{-x}$ no es aleatorio.}

\begin{definition}[c.e]
    $X$ es c.e (computablemente enumerable) si $\exists f: \mathbb{N} \to
    \mathbb{N}$ tal que $Im(f) = X$. Es decir, $f$ enumera los elementos de $X$
    en algún orden que no se puede elegir.
    \begin{example}
        $X = \{2, 5\}$. Puedo tomar $f(1) = 5, f(2) = 2$ o $f'(1) = 2, f'(2) = 5$.
    \end{example}
    \begin{example}
        $X = \{ n \mid n \text{ es impar} \} = \{1, 3, 5, 7, \dots \}$

        Puedo enumerarlos en cualquier orden,
        
        $$f(1) = 7, f(2) = 17, f(3) = 37\dots,$$
        
        y la función característica es

        $$X = 101010 \dots$$
    \end{example}
\end{definition}
\begin{definition}[Función característica]
    Puedo codificar un conjunto con una sucesión de bits correspondientes a su
    función característica.

    \[
        X = b_1 b_2 b_3 b_4 b_5 \dots
    \]

    con

    \[
        b_i =
            \begin{cases}
                1 & \text{si } i \in X \\
                0 & \text{sino.}
            \end{cases}
    \]

    Por ejemplo
    \[
        X = \{1, 4, 10\}
        =
        \underset{1}{1}\
        \underset{2}{0}\
        \underset{3}{0}\
        \underset{4}{1}\
        \underset{5}{0}\
        \underset{6}{0}\
        \underset{7}{0}\
        \underset{8}{0}\
        \underset{9}{0}\
        \underset{10}{1}
    \]
\end{definition}
\begin{observation*}
    La expansión en base 2 de $\alpha$ es la codificación en bits de $X$. Por ejemplo, con $X = \{1, 4,
    10\}$,

    $$\alpha = 2^{-1} + 2^{-4} + 2^{-10} = 0.5634765625 = (0.1001000001)_2.$$
\end{observation*}

\begin{proof}[Dem]
    Para ver que $\alpha$ no es aleatorio voy a dar un test de Martin-Löf
    $(V_i)_{i > 0}$ tal que $\alpha \in \bigcap_{i > 0} V_i$. Veamos con
    ejemplos como serían los primeros dos conjuntos:

    \begin{itemize}
        \item Supongo que $f(1) = 3$. Luego sabemos que $3 \in X$, y por lo
        tanto aparecerá en la tercera posición en su codificación en bits,
        
        \[
            X = b_1\ b_2\ 1\ b_3\ b_4\ b_5 \dots.
        \]
        
        Defino $V_1$ tal que los primeros 3 dígitos de la expansión de
        $\alpha$ aparezcan.

        \[
            V_1 = \{ 
                B_{b_1 b_2 1}
                \mid b_1, b_2 \in \{0, 1\}
            \},
        \]

        con $B_s$ las palabras infinitas que comienzan por $s$. Y su medida es
        correcta,

        \[
            \measure{V_1}
                = 2^2 \times \measure{B_{111}}
                = 2^2 \times 2^{-3} = 1/2
                \leq 2^{-1}
        \]

        \item Supongo que $f(2) = 10$, sabemos que 
        \[
            X = b_1\ b_2\ 1\ b_3\ \dots b_9\ 1\ b_{11}\ \dots.
        \]
        Defino

        \[
            V_2 = \{ 
                B_{b_1 \dots b_{10}}
                \mid b_3 = b_{10} = 1 \text{ y los demás } b_j \in \{0, 1\}
            \}.
        \]

        Su medida también es correcta,

        \[
            \measure{V_2}
                = 2^{10 - 2} \times \measure{B_{s \text{ con } |s| = 10}}
                = 2^{10 - 2} \times 2^{-10} = 2^{-2}
                \leq 2^{-2}.
        \]
    \end{itemize}

    En general, si $f(i) = x$, defino $V_i$ de la siguiente manera:

    \[
        V_i = \{ 
            B_{b_1 \dots b_{x}}
            \mid b_{f(k)} = 1 \text{ con } k \leq i \text{ y el resto } b_j \in \{0, 1\}
        \},
    \]

    luego

    \begin{align*}
        \measure{V_i}
            &= 2^{x - i} \times \measure{B_{s \text{ con } |s| = x}} \\
            &= 2^{x - i} \times 2^{-x} \\
            &= 2^{\cancel{x} - i - \cancel{x}} \\
            &= 2^{-i} \leq 2^{-i}
    \end{align*}

    y se que $\alpha \in \bigcap_{i \geq 1} V_i$ pues cada $V_i$ va refinando
    segmentos iniciales más grandes de $\alpha$. Por lo tanto, como existe un
    test de Martin-Löf que contiene a $\alpha$, $\alpha$ no es aleatorio.
\end{proof}

\subsection*{12.3}

\textit{El número $\Omega_0$ que resulta de anteponer mil 0s delante de $\Omega$ es aleatorio.}

\begin{lemma}\label{lemma:12.3}
    Como $\Omega_0 = 0^{1000} \Omega$, entonces 
    
    \begin{align*}
        \first{j}{\Omega_0}
        &= \first{j}{(0^{1000} \Omega)} \\
        &= 0^{1000} \first{j - 1000}{\Omega} \\
        &= 0^{1000} \first{i}{\Omega} && (i = j - 1000)
    \end{align*}

    A partir de $\first{j}{\Omega_0}$ puedo calcular $\first{i}{\Omega}$
    trivialmente.
\end{lemma}

\begin{proof}[Dem]
    Sea $g: \mathbb{N} \to \Sigma^*$ una enumeración de los programas que se
    detienen, $\alpha_m = \sum_{j = 1}^{m} 2 ^ {-|g(j)|}$ una aproximación de
    $\Omega$ de $m$ pasos. Defino el programa $p$,

    \[
        p = b_1 b_2 \dots b_c \first{j}{\Omega_0}^*
    \]

    donde $\first{j}{\Omega_0}^*$ es una subrutina que es el programa elegante
    (el más corto) que computa $\first{j}{\Omega_0}$ y $b_1 b_2 \dots b_c$
    realiza los siguientes pasos,

    \begin{enumerate}
        \item[0.] Computa $\first{j}{\Omega_0}$ con
        $\first{j}{\Omega_0}^*$.
        \item Computa $\first{i}{\Omega}$ en base a
        $\first{j}{\Omega_0}$ (Lema \ref{lemma:12.3}).
        \item $m = 1$. Mientras ($\alpha_m \leq \first{i}{\Omega}$), $m = m + 1$.
        \item Sea $O = \{ U(g(j)) \mid 1 \leq j \leq m \}$ los outputs de los
        programas que se detienen que aportan a la aproximación de $\Omega$.
        \item Sea $s$ la cadena más chica lexicográficamente tal que $s \notin O$.
        \item Return $s$.
    \end{enumerate}

    Veamos que $|s^*| > i$, la longitud del programa elegante que computa $s$
    (su complejidad) es más chica que $i$. Para demostrarlo supongamos lo
    contrario, que $|s^*| \leq i$

    \begin{align*}
        \Omega 
        &> 2^{-|s^*|} + \alpha_m
            &&\text{($\Omega$ tiene infinitos aportes)}\\
        &> 2^{-|s^*|} + \first{i}{\Omega}
            &&\text{(porque $\alpha_m > \first{i}{\Omega}$)}\\
        &\geq 2^{-i} + \first{i}{\Omega}
            &&\text{(sup. $|s^*| \leq i$)}\\
        &\geq \Omega
            &&\text{($\first{i}{\Omega}$ contiene exactamente los primeros $i$ bits de $\Omega$)}
    \end{align*}

    Y llegamos a $\Omega > \Omega$ que es un absurdo. Por lo tanto, $|s^*| > i$.
    
    Por otro lado, como $p$ es un programa (no muy bueno) que computa $s$, se que

    \begin{equation}\label{eq:comp-2}
         K(s) \leq c + |\first{j}{\Omega_0}^*|
    \end{equation}

    donde $c = |b_1 \dots b_c|$.

    Juntando,

    \begin{align*}
        i
            &< |s^*|\\
            &= K(s)\\
            &\leq c + |\first{j}{\Omega_0}^*|
                &&\text{por (\ref{eq:comp-2})} \\
            &= c + K(\first{j}{\Omega_0}) \\
            &\iff K(\first{j}{\Omega_0}) > i - c = j - 1000 - c
    \end{align*}

     y tomando $c' = 1000 -
c$, llegamos a

    \[
        K(\first{j}{\Omega_0}) > j - c',
    \]
    
    que es la definición de aleatoriedad de Chaitin. Por lo tanto, $\Omega_0$
    es aleatorio.

\end{proof}

\newpage
\subsection*{12.4}

\textit{Demostrar que $\alpha = \sum_{\text{palabra s}} 2^{-K(s)}$ es
computablemente aproximable desde abajo y aleatorio.}

\begin{proposition*}[$\alpha$ es aproximable desde abajo]
    Primero, observo que $\alpha$ es la probabilidad de que un programa sea
    elegante. Defino el conjunto

    \[
        S(t) = \left\{
            \begin{gathered}
            \text{programas que terminan en menos de } t \\
            \text{ pasos y son candidatos a ser elegantes}
            \end{gathered}
        \right\}.
    \]

    Si un programa $p$ está en $S(t)$ y no está en $S(t + 1)$, es porque hay
    otro programa $p'$ tal que $|p'| < |p|$ (que tarda más pasos en terminar,
    pero es más corto). Con esto puedo definir

    \[
        \alpha_t = \sum_{s \in S(t)} 2^{-|s|}.
    \]

    Como en cada paso $t$ los programas se hacen más cortos, $\alpha_t <
    \alpha_{t+1} < \alpha$ (pues son potencias negativas). Por lo tanto, alpha
    se puede aproximar computacionalmente de forma \textit{estrictamente}
    creciente desde abajo.
\end{proposition*}

\begin{proof}[Dem](12.4)
    Sea $\alpha_t$ una aproximación por abajo estrictamente creciente de $t$
    pasos de $\alpha$. Defino el programa $p$,

    \[
        p = b_1 b_2 \dots b_c \first{i}{\alpha}^*,
    \]

    donde $\first{i}{\alpha}^*$ es una subrutina que contiene el programa
    elegante (el más corto) que computa $\first{i}{\alpha}$ y $b_1 b_2
    \dots b_c$ realiza los siguientes pasos,

    \begin{enumerate}
        \item Computa $\first{i}{\alpha}$ en base a
        $\first{i}{\alpha}^*$.
        \item $t = 1$. Mientras ($\alpha_t \leq \first{i}{\alpha}$), $t = t + 1$.
        \item Sea $O = \{ U(s) \mid s \in S(t) \}$ los outputs de los
        programas candidatos a elegantes que aportan a la aproximación de
        $\alpha$.
        \item Sea $s$ la cadena más chica lexicográficamente tal que $s \notin O$.
        \item Return $s$.
    \end{enumerate}

    Veamos que $|s^*| > i$, la longitud del programa elegante que computa $s$
    (su complejidad) es más chica que $i$. Para demostrarlo supongamos lo
    contrario, que $|s^*| \leq i$

    \begin{align*}
        \alpha 
        &> 2^{-|s^*|} + \alpha_t
            &&\text{($\alpha$ tiene infinitos aportes)}\\
        &> 2^{-|s^*|} + \first{i}{\alpha}
            &&\text{(porque $\alpha_t > \first{i}{\alpha}$)}\\
        &\geq 2^{-i} + \first{i}{\alpha}
            &&\text{(sup. $|s^*| \leq i$)}\\
        &\geq \alpha.
            &&\text{($\first{i}{\alpha}$ contiene exactamente los primeros $i$ bits de $\alpha$)}
    \end{align*}

    Y llegamos a $\alpha > \alpha$ que es un absurdo. Por lo tanto, $|s^*| > i$.
    
    Por otro lado, como $p$ es un programa (no muy bueno) que computa $s$, se que

    \begin{equation}\label{eq:comp}
         K(s) \leq c + |\first{i}{\alpha}^*|
    \end{equation}

    donde $c = |b_1 \dots b_c|$.

    Juntando,

    \begin{align*}
        i
            &< |s^*|\\
            &= K(s)\\
            &\leq c + |\first{i}{\alpha}^*| &&\text{por (\ref{eq:comp})}\\
            &= c + K(\first{i}{\alpha})
    \end{align*}

    $\iff K(\first{i}{\alpha}) > i - c$, que es la definición de
    aleatoriedad de Chaitin.
    
    $\therefore \alpha$ es aleatorio.
\end{proof}

\section*{Ejercicio 13}

\textit{Dar un test de Martin Löf que contenga a los números decimales cuya expansión decimal no contiene el 7.}

\begin{proof}[Dem]

    \begin{lemma}[Contrarecíproco del ej. 9]\label{lemma:9-contrarrep}
        
        \begin{equation*}
            \aunderbrace[l1r]{
                \left(
                    \begin{gathered}
                        \text{$x$ es Martin-Löf} \\
                        \text{aleatorio}
                    \end{gathered}
                    \Rightarrow
                    x \text{ es normal.}
                \right)
            }_{\text{Ejercicio 9}}
            \implies
            \left(
                x \text{ no es normal.}
                \Rightarrow
                \begin{gathered}
                    \text{$x$ no es} \\
                    \text{Martin-Löf}\\
                    \text{aleatorio}
                \end{gathered}
            \right).
        \end{equation*}
        
    \end{lemma}

    Sea $x$ un número cuya expansión decimal no contiene el 7. Como la
    frecuencia de todos los dígitos no es la misma, no es normal. Por lo tanto,
    por el Lema \ref{lemma:9-contrarrep} tampoco es Martin-Löf aleatorio y por
    definición existirá un test en el que esté contenido.
\end{proof}

\textit{Perdón Vero por resolverlo de esta manera, pero ya me estaba extendiendo mucho con el plazo de entrega!}

\end{document}